\section*{Опыт}

\experience
{Сентябрь 2022\;--\;н. в.}
{Фронтенд-разработчик}
{АО <<АКРИХИН>>}
{Разработка и поддержка платформы для заказа препаратов у производителя <<Фарм-Заказ>> и внутренней CRM-системы.}

\experience
{Январь 2022\;--\;Июль 2022}
{Фронтенд-разработчик}
{Платформа для поступающих <<Almater>> (стартап)}
{
    \begin{itemize}[topsep=0pt]
        \item Разработал интерфейс по готовым макетам в команде из 6 человек.
        \item Проводил ревью кода второго фронтенд-разработчика. Вносил исправления, давал рекомендации по улучшению кода. Это позволило повысить стабильность работы и качество кода, упростить его поддержку.
        \item Реализовал системы авторизации и регистрации пользователей, организовал разделение по ролям и правам доступа (администратор, университет, факультет, поступающий).
        \item Создал многоступенчатую форму подачи заявки в вуз.
    \end{itemize}
}

\experience
{Сентябрь 2021\;--\;Январь 2022}
{Фронтенд-разработчик}
{ФГБУК <<Росконцерт>> (аутсорс)}
{
    \begin{itemize}[topsep=0pt]
        \item Разработал по техническому заданию и поддерживал \visiblelink{https://rosconcert.ru/}{обновлённый сайт} для организатора мероприятий по всей России.
        \item Обеспечил автоматическую генерацию и последующие обновления страниц по данным из WordPress на двух языках.
        \item Разработал режим для слабовидящих согласно международным рекомендациям и требованиям законодательства Российской Федерации.
        \item Осуществлял поиск удобных форматов хранения и редактирования информации, согласовывал их с контент-менеджером и успешно реализовывал.
    \end{itemize}
}

\experience
{Май 2021\;--\;Сентябрь 2021}
{Фронтенд-разработчик}
{Департамент информационных технологий Москвы (проект)}
{
    \begin{itemize}[topsep=0pt]
        \item После победы на хакатоне разработал интерфейс картографического сервиса для поиска коммерческой недвижимости в Москве.
        \item Реализовал систему поисковых фильтров и слоёв карты.
        \item Создал полнофункциональную мобильную версию.
        \item Оптимизировал отрисовку тысяч маркеров на карте. Благодаря этому карта не <<тормозит>>, а маркеры можно легко рассмотреть.
    \end{itemize}
}